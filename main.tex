\documentclass[master,phil,fulltime,electronic,cenhdr,normaltoc]{buptthesis} % 学硕电子版
%\documentclass[master,phil,fulltime,print,cenhdr,normaltoc]{buptthesis} % 学硕印刷版
%\documentclass[master,eng,fulltime,electronic,cenhdr,normaltoc]{buptthesis} % 专硕电子版
%\documentclass[master,eng,fulltime,print,cenhdr,normaltoc]{buptthesis} % 专硕印刷版
%\documentclass[phd,phil,fulltime,electronic,cenhdr,normaltoc]{buptthesis} % 学博电子版
%\documentclass[phd,phil,fulltime,print,cenhdr,normaltoc]{buptthesis} % 学博印刷版
%\documentclass[phd,eng,fulltime,electronic,cenhdr,normaltoc]{buptthesis} % 专博电子版
%\documentclass[phd,eng,fulltime,print,cenhdr,normaltoc]{buptthesis} % 专博印刷版


% !UPDATE 20231210: add to remove `normaltoc` to remove table content in the toc

% 点子版和印刷版的区别
% 1. 电子版是没有跳偶数页,所有内容都是直接连起来的
% 2. 印刷版部分内容跳偶数页开头的,生成的PDF直接双页打印即可

\usepackage{graphicx}
\usepackage{bbding}
\usepackage{booktabs}
\usepackage{bicaption}
\usepackage{float}
\usepackage{gbt7714}
\usepackage{fontspec}

% 定义使用的函数/宏
\DeclareMathOperator{\Tr}{Tr}

% 正文当中的其他定义
\setmainfont{Times New Roman}
\setmathfont{XITS Math}

\BUPTthesiscntitlepage{%
	confidential = {公开}, % 目前只支持公开
	title = {文章 \\ 标题 },  % 本章标题
	studentid = {2021123456}, % 学号
	name = {你的名字}, % 姓名
	major = {专业名称}, % 专业名称
	supervisor = {你的导师}, % 导师姓名
	institute = {你的学院} % 学院名称
}

%% \quad 相当于一个汉字的宽度,同时也相当于~~~~

\BUPTdateOptions{%
    pageYear = {2024}, % 封面上面的年
    pageMonth = {5}, % 封面上面的月
    pageDay = {30}, % 封面上面的天
    dissertationYear = {2024}, % 答辩时间的年
    dissertationMonth = {5}, % 答辩时间的月
    dissertationDay = {30} % 答辩时间的天
}

\BUPTthesisentitlepage{%
	confidentialEN = {Public},  % 目前只支持公开
	titleEN = {Thesis \\ Title}, % 本章标题
	nameEN = {Your Name}, % 姓名
	majorEN = {Your Major}, % 专业名称
	supervisorEN = {Your Boss}, % 导师姓名
	instituteEN = {Your Institute} % 学院名称
}

\BUPTdissertationcommitteepage{%
	committeeChairName = {教授1}, % 主席名称
	committeeChairTitle = {教授}, % 主席职称
	committeeChairOrgan = {北京邮电大学}, % 主席单位
	committeeMemberFirstName = {名称2}, % 委员1名称
	committeeMemberFirstTitle = {教授},
	committeeMemberFirstOrgan = {北京邮电大学},
	committeeMemberSecondName = {名称3},
	committeeMemberSecondTitle = {教授},
	committeeMemberSecondOrgan = {北京邮电大学},
	committeeSecretaryName = {名称4}, % 秘书名称
	committeeSecretaryTitle = {副教授},
	committeeSecretaryOrgan = {北京邮电大学},
}
% 最多支持四个委员,分别从First->Fourth,不指定参数就为空

\begin{document}

\makechinesetitle
\makeenglishtitle
\makecommitteelist
\makestatement

\frontmatter
\include{Chapter/Chapter_cnabstract} % 中文摘要
\include{Chapter/Chapter_enabstract} % 英文摘要
\tableofcontents
\begin{nomenclature}
 \begin{table*}[htbp]
   \centering
   \renewcommand\arraystretch{1.5}
   \begin{tabular}{>{\centering\arraybackslash}m{4cm} >{\centering\arraybackslash}m{4cm} >{\centering\arraybackslash}m{4cm}}
      \toprule
      符\quad 号   & 代表意义                              & 单\quad 位  \\  
      $A$          & 截面积                                & $m^{2}$ \\
      \bottomrule
    \end{tabular}
 \end{table*}
\end{nomenclature}
 % 符号表
% \listoffigures % NOTE:20230103旧模板
% \listoftables % NOTE:20230103旧模板

\mainmatter
\chapter{例子展示}

一些基本使用的例子

\section{一级标题}
\subsection{二级标题}
\subsubsection{三级标题}

\section{图表和表格}

目前只实现了最基本的应用,没有实现官方word模板中使用题注的方式介绍详细图表信息。引用图\ref{fig:bupt_logo}和表\ref{tab:bupt_table}信息。

参考文献\cite{ddpg}。

\begin{figure}[htbp]
  \centering
  %\vspace{1 cm} % 使用该指令调整上下间距
  \includegraphics[width=0.5\textwidth]{resources/logo.png}
  \bicaption{中文标题}{English Caption} % 使用中英文标题
  %\caption{中文标题} % 只使用中文标题
  \label{fig:bupt_logo}
  %\vspace{1 cm} % 使用该指令调整上下间距
\end{figure}

\begin{table}[htbp]
  \centering
  %\vspace{1 cm} % 使用该指令调整上下间距
  \bicaption{中文标题}{English Caption}
  \renewcommand\arraystretch{1.5}
  \begin{tabular}{>{\centering\arraybackslash}m{4cm} >{\centering\arraybackslash}m{4cm} >{\centering\arraybackslash}m{4cm}}
    \toprule
    符\quad 号 & 代表意义 & 单\quad 位 \\
    $A$      & 截面积  & $m^{2}$  \\
    \bottomrule
  \end{tabular}
  \label{tab:bupt_table}
  %\vspace{1 cm} % 使用该指令调整上下间距
\end{table}


\section{公式}

\begin{equation}
  a + b = c
\end{equation}


\section{算法}

目前算法的标题是中文。% 如果需要变成英文Algorithm,在`buptthesis.cls'文件当中注释掉`\renewcommand{\ALG@name}{算法}`这个代码。
算法模块默认导入`algorithm'和`algorithmic'库,如果使用`algorithm2e'会发生冲突。引用算法\ref{alg:sum}信息。

\begin{algorithm}[htbp]
    \caption{algorithm of SUM}
    \label{alg:sum}
    \renewcommand{\algorithmicrequire}{\textbf{Input:}}
    \renewcommand{\algorithmicensure}{\textbf{Output:}}
    \begin{algorithmic}[1]
        \REQUIRE $A$, $B$, $C$  %%input
        \ENSURE EEEEE    %%output
        
        \STATE  AAAAA
        \WHILE{$A=B$}
            \STATE BBBBB
        \ENDWHILE
        
        \FOR{each $i \in [1,10]$}
            \IF {$C = 0$}
                \STATE CCCCC
            \ELSE
                \STATE 使用中文
            \ENDIF
        \ENDFOR
        
        \RETURN EEEEE
    \end{algorithmic}
\end{algorithm}

\bibliographystyle{gbt7714-numerical}
{\zihao{5}
	\bibliography{Bib/thesis}
}

\backmatter
\include{Chapter/Chapter_abbreviation} % 缩写附录,暂时还没有写这里
\include{Chapter/Chapter_acknowledgement} % 致谢
\begin{publication}
\definecolor{Maroon}{HTML}{701112}
\newcommand{\IF}[1]{\textcolor{Maroon}{IF = #1}}
本人攻读学位期间共发表论文X篇,其中,第一作者X篇,学生一作X篇,第二作者X篇,其他参与论文共X篇:
\begin{enumerate}
\item AuthorA, Author B, Author C, Paper Title, \emph{Conference/Transactions}, May. 2021, pp. 1-15. (\IF{999.99})
\end{enumerate}

发明专利0项


\end{publication}
 % 创新成果

\end{document}
